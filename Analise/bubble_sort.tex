A complexidade temporal do Bubble Sort é classificada como \(O(n^2)\) em todos os três casos: melhor, médio e pior, devido à ausência de uma condição de parada precoce no algoritmo.

\begin{table}[htbp]
\centering
\begin{tabular}{|c|c|c|c|}\hline
    
    \textbf{Linhas} & \textbf{Algoritmo} & \textbf{Custo} & \textbf{Vezes}
    \\\hline
1 & para i = 1 ate comprimento-1 & C1 & n-1\\
    2 &\quad para j = 1 ate comprimento-i & C2 & \sum\limits_{1}^{\mbox{n-1}{}}T_j\\
    3 &\quad se A[j] $>$ A[j+1] então & C3 & \sum\limits_{1}^{\mbox{n-1}{}}T_j-1\\
    4 &\quad swap(A[j], A[j+1]) & C4 & \sum\limits_{1}^{\mbox{n-1}{}}T_j-1\\
    \hline
\end{tabular}
\caption{Tabela de complexidade bubble.}
\end{table}

A fórmula geral para a complexidade temporal do Bubble Sort pode ser expressa como:
\begin{equation}
T(n) = C_1(n-1) + C_2\sum_{j=1}^{n-1} T_j + C_3\sum_{j=1}^{n-1} (T_j - 1) + C_4\sum_{j=1}^{n-1} (T_j - 1)
\end{equation}

Onde, 
\[\sum_{j=1}^{n-1} T_j = \frac{n(n-1)}{2}\]
resultante da soma dos \(n-1\) primeiros números naturais, sendo uma série aritmética.

Ao desenvolver a fórmula geral, temos:
\begin{align}
T(n) &= C_1n - C_1 + C_2\left(\frac{n^2 - n}{2}\right) + C_3\left(\frac{n^2 - 3n + 2}{2}\right) + C_4\left(\frac{n^2 - 3n + 2}{2}\right) \\
&= (C_2 + C_3 + C_4)\frac{n^2}{4} + \left(C_1 - C_2 - \frac{3(C_3 + C_4)}{2}\right)n - (C_1 + C_3 + C_4)
\end{align}

Portanto, reorganizando os termos, podemos concluir que a função de complexidade é quadrática, assim, \(T(n) = O(n^2)\).