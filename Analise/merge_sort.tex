A fórmula recorrente para o tempo de execução do Merge Sort é dada por:

\begin{equation}
    T(n) = 2T\left(\frac{n}{2}\right) + \Theta(n)
\end{equation}

Nesta equação, o termo \( 2T\left(\frac{n}{2}\right) \) representa o tempo necessário para ordenar dois subconjuntos de tamanho \( \frac{n}{2} \). O termo \( \Theta(n) \) denota o tempo necessário para mesclar esses subconjuntos, ou seja, é o tempo gasto pela função de mesclagem (merge).

\begin{table}[H]
\centering
\caption{Tabela de custos do Merge}
\label{tab:custos_merge}
\begin{tabular}{|c|l|c|c|}
\hline
\textbf{Linhas} & \textbf{Algoritmo} & \textbf{Custo} & \textbf{Vezes} \\ \hline
1 & \texttt{while i \( \leq \) q+1 and j \( \leq \) r+1} & \( C1 \) & \( n \) \\
2 & \texttt{if V[i] \( \leq \) V[j]} & \( C2 \) & \( n-1 \) \\
3 & \texttt{w[k++] = V[i++]} & \( C3 \) & \( n-1 \) \\
4 & \texttt{else w[k++] = V[j++]} & \( C4 \) & \( n-1 \) \\
5 & \texttt{while i \( < \) q} & \( C5 \) & \( n \) \\
6 & \texttt{w[k++] = V[i++]} & \( C6 \) & \( n-1 \) \\
7 & \texttt{while j \( < \) r} & \( C7 \) & \( n \) \\
8 & \texttt{w[k++] = V[j++]} & \( C8 \) & \( n-1 \) \\
9 & \texttt{for i=p; i \( < \) r; ++i} & \( C9 \) & \( n \) \\
10 & \texttt{V[j] = w[i-p]} & \( C10 \) & \( n-1 \) \\ \hline
\end{tabular}
\end{table}


Para resolver esta equação recorrente, podemos empregar o método de Akra-Bazzi ou o Teorema Mestre. Neste caso, escolheremos o último.

Conforme o Teorema Mestre, os parâmetros são:

\begin{align}
    a &= 2, \\
    b &= 2, \\
    f(n) &= \Theta(n)
\end{align}

Adicionalmente, temos:

\begin{equation}
    \log_b a = \log_2 2 = 1
\end{equation}

Dado que \( f(n) = \Theta(n^{\log_b a}) = \Theta(n) \), a equação de recorrência se enquadra no segundo caso do Teorema Mestre. Portanto, temos:

\begin{equation}
    T(n) = \Theta(n^{\log_b a}\log n) = \Theta(n \log n)
\end{equation}

Ou, de forma equivalente, \( T(n) = O(n \log n) \). Esta complexidade é mantida tanto para o melhor caso, quanto para o caso médio e o pior caso.
