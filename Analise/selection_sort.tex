A complexidade do Selection Sort é a mesma, $O(n^2)$, para os três casos: melhor, pior e médio caso, pois seu algoritmo não contém condição de parada.

Fórmula Geral \ref{eq:formula_geral}:
\begin{equation}
T(n) = C_1n + C_2(n-1) + C_3\left(\sum_{j=2}^{n} T_j \right) + C_4\left(\sum_{j=2}^{n} (T_j - 1) \right) + C_5\left(\sum_{j=2}^{n} (T_j - 1) \right) + C_6(n-1)
\label{eq:formula_geral}
\end{equation}

\begin{table}[H] 
\centering
\begin{tabular}{|c|c|c|c|}
    \hline
    \textbf{Linhas} & \textbf{Algoritmo} & \textbf{Custo} & \textbf{Vezes} \\\hline
    1 & para \(i = 1\) até \textit{comprimento} - 1 & \(C_1\) & \(n\) \\
    2 & \quad menor = i & \(C_2\) & \(n-1\) \\
    3 & \quad para \(j = i + 1\) até \textit{comprimento} & \(C_3\) & \(\sum_{j=2}^{n} T_j\) \\
    4 & \quad se \(A[j] < A[\text{menor}]\) & \(C_4\) & \(\sum_{j=2}^{n} (T_j - 1)\) \\
    5 & \quad menor = j & \(C_5\) & \(\sum_{j=2}^{n} (T_j - 1)\) \\
    6 & \textit{swap}(A[j], A[menor]) & \(C_6\) & \(n-1\) \\
    \hline
\end{tabular}
\caption{Tabela de complexidade do Selection Sort.}
\label{tab:complexidade_selection}
\end{table}



\paragraph{PIOR CASO:}
No pior caso, temos a seguinte expressão para \(T(n)\):

\begin{align}
T(n) &= C_1n + C_2(n-1) + C_3\left( \frac{n^2 + n - 2}{2} \right) + C_4\left( \frac{n^2 - n}{2} \right) + C_5\left( \frac{n^2 - n}{2} \right) + C_6(n - 1) \nonumber \\
&= C_1n+C_2n-C_2+C_3\frac{n^2}{2} + C_3n - C_3 + C_6n-C_6 \nonumber \\
&= (C_3) \cdot n^2 + (C_1 + C_2 + C_3 + C_6) \cdot n - (C_2 + C_3 + C_6) \quad \text{(Função Quadrática)}
\end{align}

Portanto, a complexidade no pior caso é \(O(n^2)\).

\paragraph{MELHOR CASO:}
No melhor caso, a linha \(C_5\) não é executada, pois a condição da linha \(C_4\) é falsa. Desta forma, temos:

\begin{align}
T(n) &= C_1n + C_2(n-1) + C_3\left( \frac{n^2 + n - 2}{2} \right) + C_4\left( \frac{n^2 - n}{2} \right) + C_6(n - 1) \nonumber \\
&= C_1n + C_2n - C_2 + C_3\frac{n^2}{2} + C_3n - C_3 + C_6n-C_6 \quad \text{(Função Quadrática)}
\end{align}

Assim, a complexidade no melhor caso também é \(O(n^2)\).

\paragraph{CASO MÉDIO:}
Para o caso médio, dividimos os somatórios já desenvolvidos por 2, ou multiplicamos por \(\frac{1}{2}\). 

\begin{equation}
T(n) = C_1n + C_2(n-1) + C_3\left( \frac{n^2 + n - 1}{2} \right) +C_4\left(\frac{1}{4}n(4n-1)\right) +C_5\left(\frac{1}{4}n(4n-1)\right) +C_6(n-1)
\end{equation}

Aplicando a distributiva e agrupando os valores, temos uma função quadrática:

\begin{equation}
T(n) = (C_3) \cdot n^2 + (C_1 + C_2 + C_3 + \frac{1}{4}C_4 + \frac{1}{4}C_5 + C_6) \cdot n - (C_2 + \frac{1}{2}C_3 + \frac{1}{4}C_4 + \frac{1}{4}C_5 + C_6)
\end{equation}

Logo, a complexidade no caso médio é \(O(n^2)\).