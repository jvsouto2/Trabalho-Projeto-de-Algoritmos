O algoritmo Shell Sort é conhecido por sua complexidade de tempo não possuir uma expressão fechada definida\cite{sedgewick1986shell}. A complexidade deste algoritmo tem as seguintes características:
A complexidade do Shell Sort é altamente dependente da escolha da sequência de incrementos. Diferentes sequências produzem diferentes complexidades de tempo. A determinação de uma sequência ótima de incrementos é um problema em aberto e várias sequências têm sido propostas com o intuito de otimizar o desempenho do algoritmo.
Uma análise exata da complexidade de tempo do Shell Sort ainda não foi alcançada. Embora testes empíricos e análises parciais sugiram que a complexidade pode ser, em alguns casos, quase linear(\( T(n) = O(n^{1.25}) \) e \( T(n) = O(n (\ln n)^2) \)), uma descrição analítica precisa em termos de notação Big-O para uma sequência de lacunas arbitrária não foi estabelecida.
