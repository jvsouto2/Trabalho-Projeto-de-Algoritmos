\qquad Este artigo realiza um estudo comparativo sobre quatro algoritmos de ordenação: Insertion Sort, Selection Sort, Bubble Sort, Shell Sort, Merge Sort, Quick Sort(Esse será testado usando 4 métodos diferentes de implementação, cada um usando um valor para o pivô) e o HeapSort e suas operações de Fila de Prioridade. A proposta central do estudo é dissecar as características, a eficácia e as limitações de cada algoritmo, fornecendo uma visão sobre o seu comportamento em diferentes contextos e condições de entrada, além da definição do que é cada algoritmo, será mostrado exemplos para que fique mais fácil a compreensão dos mesmos. Na busca por compreender as variações de cada método de ordenação, o artigo aborda a implementação e análise de cada algoritmo, avaliando seu desempenho através de diferentes tamanhos de conjuntos de dados, que variam de n=10 a 1.000.000, e em diferentes estados de ordenação, como Crescente, Decrescente e Aleatória. Esta análise permite discernir o comportamento e a eficiência relativa de cada algoritmo, com o intuito de elucidar cenários de aplicação ideais para cada um deles. Com isso, será estudado a complexidade de cada algoritmo no seu melhor, pior e médio caso, quando existir, buscando a fórmula Geral para cada uma das situações. O artigo também incorpora gráficos comparativos e tabelas de complexidade para ilustrar visualmente as diferenças de desempenho e complexidade entre os algoritmos, servindo como um guia visual para a compreensão e seleção apropriada de algoritmos de ordenação.

\[...\]