\begin{table}[h]
    \centering
    \caption{Comparação do Tempo do Seletion Sort}
    \begin{tabular}{|c|c|c|c|c|c|c|}
        \hline
        Tamanho de Entrada & 10 & 100 & 1000 & 10000 & 100000 & 1000000 \\
        \hline
        Crescente & 0.000000 & 0.000000 & 0.001000 & 0.103000 & 10.177000 & 1089.101000 \\
        \hline
        Decrescente & 0.000000 & 0.000000 & 0.000000 & 0.098000 & 9.769000 & 1046.444000 \\
        \hline
        Aleatória & 0.000000 & 0.000000 & 0.000000 & 0.102000 & 10.196000 & 1055.879000 \\
        \hline
    \end{tabular}
    \label{tab:comparacaoinsert}
\end{table}

Para a elaboração da tabela (Tabela \ref{tab:comparacaoinsert}) e do gráfico (Figura \ref{fig:select1}), foi desenvolvido um algoritmo em C com o intuito de gerar arquivos de diferentes tamanhos de entrada, 10, 100, 1.000, 10.000, 100.000 ou 1.000.000, contendo sequências de números, bem como seus n sucessores. Essas sequências puderam ser geradas em ordem Crescente, Decrescente ou Aleatória. Os resultados obtidos revelam características do funcionamento do Selection Sort. Este algoritmo demonstra consistência em sua performance, apresentando resultados similares em diferentes cenários de entrada. O Selection Sort tem uma performance homogênea e previsível, independentemente de o conjunto de dados estar ordenado, parcialmente ordenado ou em ordem inversa.

\begin{figure}[h!]
    \centering
    \includegraphics[width = 10cm]{Imagens/Selection Sort/Captura de Tela 2023-09-27 às 18.27.08.png}
    \caption{Gráfico de tempo do algoritmo Selection Sort.}
    \label{fig:select1}
\end{figure}