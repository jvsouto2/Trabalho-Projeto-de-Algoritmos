\qquad Em várias situações do nosso dia-a-dia nos deparamos com a necessidade de trabalharmos com dados/informações devidamente ordenadas, como, por exemplo, ao procurar um contato na lista telefônica, imagine como seria difícil se estes nomes não estivessem em ordem alfabética? Então não é difícil perceber que as atividades que envolvem algum método de ordenação estão muito presentes na
computação.
\par Neste trabalho, realizamos uma análise detalhada de quatros algoritmos de ordenação essenciais: Bubble Sort, Insertion Sort, Selection Sort, Shell Sort, Merge Sort, Quick Sort e o HeapSort. O comportamento destes algoritmos será examinado em várias condições de entrada de dados (10, 100, 1000, 10000, 100000 e 1000000), tratando sobre suas complexidades computacionais no melhores, médios e piores casos. A eficiência desses algoritmos é crucial, pois uma escolha inadequada pode levar a um desempenho insatisfatório, especialmente em grandes conjuntos de dados\cite{cormen2002}.
\par A ênfase é dada ao estudo da complexidade computacional, que é um pilar central na avaliação da eficiência de um algoritmo. A complexidade computacional nos permite entender o comportamento de um algoritmo em termos de tempo de execução e uso de recursos, dependendo do tamanho do conjunto de dados processado\cite{silva2023analise}.
\par Ao analisar os algoritmos, consideramos aspectos como o número de comparações e trocas realizadas, além de como cada um lida com dados em diferentes estados de ordenação. A compreensão desses fatores é essencial para selecionar o algoritmo mais adequado para uma dada aplicação, garantindo não apenas eficiência, mas também a otimização de recursos.
\par Com este estudo, buscamos oferecer uma visão abrangente sobre a seleção e a aplicação de algoritmos de ordenação, fornecendo um guia prático para profissionais da computação no que diz respeito à escolha de métodos de ordenação em diferentes cenários.


\[...\]